\part{مقدمه}

کتابخانه‌های عمومی یک از مراکزی هستند که استفاده از پایگاه‌های داده
می‌تواند کمک شایانی در مدیریت و مرتب کردن لیست کتاب‌ها و کاربرها
کند. همچنین مدیریت کتاب‌های امانت گرفته شده و کاربرهایی که عضو
کتابخانه هستند با استفاده از پایگاه‌داده و در کل برنامه‌های کامپیوتری
بسیار کمک کننده هستند و باعث بیشتر شدن بازدهی می‌شوند.

زبان برنامه‌نویسی پایتون به دلیل سادگی سینتکس و شی‌گرا بودن، یکی از
گزینه‌های مناسب برای نوشتن این برنامه است. سادگی زبان و درعین حال شی‌گرا
بودن پایتون باعث می‌شود درحالی که نوشتن برنامه اولیه بسیار آسان و ساده
باشد، نگه‌داری برنامه برای استفاده بلند مدت از برنامه مناسب است. همچنین
وجود کتابخانه‌ها و ماژول‌های بسیار زیاد نوشته شده برای پایتون، باعث
می‌شود پایتون را گزینه مناسبی برای ساختن برنامه‌های مختلف در حوزه‌های
بسیار زیادی می‌کند.

به دلیل محبوبیت پایتون در بین برنامه‌نویس‌ها، برنامه‌های جانبی بسیار
زیادی برای کمک کردن به برنامه‌نویس ساخته شده است. یک نمونه از این برنامه‌ها
که در نوشتن کدها استفاده شد \textit{پای‌رایت}\LTRfootnote{PyRight} است که یک
\textit{ال‌اس‌پی} \LTRfootnote{LSP (Language Server Protocol)} می‌باشد. کار این
برنامه‌ها بررسی کدهای نوشته شده و شناسایی کلمات کلیدی، نوع کلمات کلیدی
و محلی که تعریف یا استفاده شده‌اند می‌باشد که کار برنامه‌نویس را هنگام
عیب‌یابی برنامه، بسیار سریعتر و پربازده‌تر می‌کنند.

ویرایشگر ویم یک برنامه \textit{تحت متن}
\LTRfootnote{TUI (Text-based User Interface)}
می‌باشد که قابلیت \textit{برنامه‌نویسی شدن}\LTRfootnote{Programable}
را دارد. این قابل برنامه‌نویسی بودن ویم باعث شده که کاربرهای این برنامه
\textit{افزونه}‌های\LTRfootnote{Plugins} بسیار زیادی برای انجام بهتر و سریعتر
کارهایشان بنویسند و بصورت عمومی برای استفاده دیگر برنامه‌نویس‌ها منتشر
کنند. بعنوان مثال قابلیت استفاده از پای‌رایت در ویم توسط یک پلاگین به
آن اضافه می‌شود. همچنین ویم قابلیت شخصی‌سازی بسیار بالایی دارد که
برنامه‌نویس‌ها می‌توانند دقیقا بسته به نیاز و علایق‌شان آن را تغییر دهند و
شخصی‌سازی کنند.

% \ThreeColumnFootnotes
\vfill\newpage

\section{زبان پایتون}

پایتون یک زبان برنامه نویسی سطح بالا و همه منظوره است. فلسفه طراحی آن
بر خوانایی کد با استفاده از \textit{تورفتگی}\LTRfootnote{Indentation} قابل توجه تأکید دارد. پایتون به
صورت \textit{تایپ پویا}\LTRfootnote{Dynamic-types} می‌باشد و دارای \textit{گاربج کالکتور}\LTRfootnote{Garbage collector} است.
این پارادایم های برنامه نویسی متعدد،
از جمله برنامه نویسی \textit{ساختاریافته}\LTRfootnote{structured}،
\textit{شی‌گرا}\LTRfootnote{Object orianted} و \textit{تابعی}\LTRfootnote{Functional} را
پشتیبانی می‌کند. به دلیل کتابخانه استاندارد جامع آن، اغلب به عنوان یک
زبان «همراه با باتری» توصیف می شود. \textit{خیدو فان روسوم}\LTRfootnote{Guido van Rossum} در اواخر دهه 1980
کار بر روی پایتون را به عنوان جانشین زبان برنامه نویسی \textit{ای‌بی‌سی}\LTRfootnote{ABC} آغاز کرد و
اولین بار در سال 1991 آن را با نام پایتون ۰.۹.۰ منتشر کرد.
پایتون 0.3 در سال 2000 منتشر شد. پایتون 0.3 که در سال 2008 منتشر شد، یک نسخه اصلی
بود که کاملاً با نسخه‌های قبلی سازگار نبود.

پایتون به‌جای ساختن تمام قابلیت‌های خود در هسته‌اش، به گونه‌ای طراحی شد که
از طریق ماژول‌ها بسیار توسعه‌پذیر باشد. این ماژولار بودن فشرده آن را به
عنوان وسیله ای برای افزودن رابط های قابل برنامه ریزی به برنامه های
موجود محبوب کرده است. این دیدگاه ون روسوم از یک زبان اصلی کوچک با یک
کتابخانه استاندارد بزرگ و یک مترجم به راحتی قابل توسعه، ناشی از
ناامیدی او از ای‌بی‌سی بود، که از رویکرد مخالف حمایت می کرد.


% \TwoColumnFootnotes

% % tmp {{{
% \begin{src}[نمونه]
% \begin{minted}{python}
% class Ui_MainWindow(object):
%     def setupUi(self, MainWindow):
%         MainWindow.setObjectName("MainWindow")
%         MainWindow.setWindowModality(QtCore.Qt.NonModal)
%         MainWindow.resize(789, 592)
%         self.centralwidget = QtWidgets.QWidget(MainWindow)
%         self.centralwidget.setObjectName("centralwidget")
%         self.gridLayout = QtWidgets.QGridLayout(self.centralwidget)
% \end{minted}
% \end{src}

% \begin{figure}[ht!]
%     \centering
%     \caption{یک عکس نمونه}
%     \includegraphics[width=0.2\textwidth]{/home/hos/pictures/test.png}
% \end{figure}

% \begin{table}[ht!]
%     \centering
%     \begin{tabular}{ccc}
%         \hline
%         \textbf{یک} & \textbf{دو} & \textbf{سه} \\
%         \hline
%         نمونه & شماره & یک \\
%         در حال & تست & شدن \\
%         \hline
%     \end{tabular}
%     \caption{یک جدول آزمایشی}
% \end{table}
% % }}}
