\renewcommand{\thepage}{\texorpdfstring{\lr{\arabic{page}}}{\arabic{page}}}

\setcounter{MyC}{1}
% \addAdadi\addAdadi\addAdadi\addAdadi\addAdadi\addAdadi
\setcounter{MyC}{1}
% \addHarfi\addHarfi\addHarfi\addHarfi\addHarfi
\setcounter{MyC}{1}
% \addTartibi\addTartibi\addTartibi\addTartibi\addTartibi

\begin{abstract}
این برنامه مربوط به مدیریت یک کتابخانه عمومی با استفاده از پایگاه‌داده‌های رابطه‌ای\LTRfootnote{SQL} است.
از قابلیت‌های این برنامه می‌توان به \textbf{مدیریت کتاب‌ها}، \textbf{مدیریت
کاربرها}، و \textbf{مدیریت کتاب‌های امانت گرفته شده}، اشاره کرد. تکنولوژی‌هایی که
برای ساخت این برنامه استفاده شده‌اند عبارتند از زبان برنامه نویسی پایتون،
کتابخانه‌های پایتون مانند پای‌کیوت\LTRfootnote{PyQt5} برای طراحی رابط کاربری گرافیکی و
مای‌سیکویل کانکتور\LTRfootnote{mysql-connector-python} برای متصل کردن برنامه به پایگاه‌داده‌ها. همچنین
از ویرایشگر متن ویم\LTRfootnote{Vim} برای نوشتن برنامه و از کیوت دیزاینر\LTRfootnote{QtDesginer} برای طراحی رابط
کاربری گرافیکی استفاده شد.
نحوه نوشتن برنامه بصورت شی‌گرا و ماژولار است. به این‌گونه که هر پنجره ساخته شده یک
کلاس کاملا جدا بوده که برای انجام وظایف محوله، متدهای خاص خود را دارند.
\end{abstract}

% \begin{multicols}{4}
\subsubsection*{کلمات کلیدی}
\begin{itemize}
    \item پایتون\LTRfootnote{Python}
    \item کیوت\LTRfootnote{QT}
    \item مای‌سیکویل\LTRfootnote{MySQL}
    \item ماژول\LTRfootnote{Module}
    \item کلاس\LTRfootnote{Class}
    \item متد\LTRfootnote{Method}
    \item رابط کاربری گرافیکی\LTRfootnote{GUI (Graphical User Interface)}
\end{itemize}
% \end{multicols}

\pagenumbering{adadi}
\vfill\newpage

\tableofcontents
\vfill\newpage

\listoffigures
\vfill\newpage

\listoftables
\vfill\newpage

\lstlistoflistings
\vfill\newpage

\vfill\newpage
\pagenumbering{arabic}
